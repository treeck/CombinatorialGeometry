If the relative speed, $c$, of two frames of reference
is the Euclidean distance, $d$, traveled in time, $t$, then $d = ct$.
And if $d_{2}$ is the distance covered in time $t$ traveling at speed, $v$,
then $d_{2} = vt$ in the second frame of reference.
\begin{multline}\label{eq:space-contraction}
d^{2}=d_{1}^{2} + d_{2}^{2}\quad\Rightarrow\quad
d_{1}^{2} =  d_{2}^{2} - d^{2}\quad\Rightarrow\\
d_{1}^{2} =  d_{2}^{2} - d^{2},
\quad d_{2}=vt, \quad d=ct \quad\Rightarrow\quad
d_{1}^{2} =  (ct)^2 - (vt)^{2}
\quad\Rightarrow\\
d_{1}^{2}=(ct)^2 - (ct(v/c))^{2}\quad\Rightarrow\quad
d_{1}^{2}=ct(1-(v/c)^{2})^{1/2},
\end{multline}
